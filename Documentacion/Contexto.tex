%!TEX root =  tfg.tex
\chapter{Contexto}

\begin{quotation}[Oceanographer]{Jacques Yves Cousteau}
The sea, once it casts its spell, holds one in its net of wonder forever.
\end{quotation}

\begin{abstract}
Vamos a dar una overview del contexto en el que se desarrolla el proyecto, incluyendo el estado actual de la industria y el porque del proyecto.
\end{abstract}

\section{Redes sociales y exploración marina}

Desde pequeño, he vivido rodeado de mar, pero es solo ahora cuando he descubierto lo mucho que me gusta el buceo,
sin embargo, este viene con una serie de retos y problemas, ¿Donde me sumerjo? ¿Hará buen clima debajo del agua? 
¿Podré llegar a la zona con mi equipo o hara mucho viento? Con este proyecto, busco dar una solucioncilla a estos problemas, 
de una forma tanto estadisitica como personal, permitiendo a las personas ver datos recopilados a lo largo de los años sobre zonas de buceo, 
su ecosistema y condiciones climatoloógicas y a la vez, darles libertad para que puedan compartir sus propias experiencias. Para que el usuario
sea el que decida si quiere guiarse por los datos o por la experiencia de otros.

\section{Aplicaciones y tecnologías similares}

Hoy en día, las aplicaciones existen a montones, para nichos sorprendentemente especificos. Scubex no es nada revolucionario,
pues varias aplicaciones de las que se inspira y apoya ya incluyen algunas de sus funcionalidades. Pero sin embargo ninguna de ellas
las incluye de forma intuitiva, centralizada y orientada unicamente a buceadores. Son 3 las aplicaciones que he podido ver las cuales son
similares a Scubex.

La primera es Dive Mate: Una aplicación que te muestra especias que has visto tú en una zona y que te da estimaciones climatológicas precisas, sin embargo, 
tanto la interfaz como la posibilidad de compartir experiencias son muy limitadas, la aplicación denota torpeza en su interfaz. Razón por la cual nuevos y antiguos bucedores podrían 
no sentirse atraidas por ella. 

La segunda es iNaturalist: Una aplicación que te permite ver, en un mapa, avistamientos de especies en esa zona, sus fotos e información de avistamientos es increiblemente
rica y útil, sin embargo no esta centralziada a los buceadores y no incluye información climatológica para el mar de la zona. Es por ello que (detallando mas adelante) usaremos su API 
para recopilar información sobre una zona y mostrarla al usuario.

La última es DiveApp: Una aplicación orientada a los avistamientos y preparación de viajes, incluye componentes de red social. Sin embargo estos estan orientados unicamente a la experiencia
por lo que no incluye ni información recopilada ni información climatológica, únicamente se basa en la experiencia de los usuarios.