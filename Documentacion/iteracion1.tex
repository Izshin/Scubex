%!TEX root =  tfg.tex
\chapter{Iteración 1: Escáner de Biodiversidad}

\begin{abstract}
Esta iteración aborda el núcleo funcional de Scubex: la capacidad de identificar especies marinas en una zona determinada mediante la orquestación de OBIS e iNaturalist. He seguido un enfoque TDD para implementar la compleja lógica de transformación geométrica y filtrado cruzado de datos científicos.
\end{abstract}

\section{Características a desarrollar}

\begin{enumerate}
\item Escáner de Especies bajo demanda (Backend). Ver Tabla \ref{tab:valorAportadoScan}.
\item Filtrado automático de microorganismos y especies sin representación visual.
\item Enriquecimiento de datos con fotografías y nombres comunes.
\end{enumerate}

\begin{table*}[htb]
	\centering
	\begin{coolTable}{p{4cm}p{\textwidth-4.5cm}}{2}
{Análisis de valor aportado: Escáner de Especies}
\textbf{Propuesta}&Implementación de un endpoint REST que agregue datos de OBIS e iNaturalist para una zona geográfica definida por radio.\\
\midrule
\textbf{Valor}&Permite al usuario descubrir fauna marina real validada científicamente con representación visual, sin conocimientos técnicos ni acceso directo a bases de datos especializadas.\\
\textbf{Coste}&Alto esfuerzo de integración. Requiere orquestar llamadas asíncronas, transformar formatos geométricos (coordenadas → WKT Polygon), gestionar inconsistencias entre APIs y aplicar filtrado de calidad.\\
\textbf{Opciones}&\textbf{Opción 1}: Usar solo OBIS reduciría el coste, pero eliminaría el valor visual (fotos) y semántico (nombres comunes), haciendo la app inutilizable para turistas/buceadores recreativos. \textbf{Opción 2}: Usar solo iNaturalist (solo observaciones ciudadanas, sin validación científica OBIS).\\
\textbf{Riesgos}&Rate limits de APIs externas (OBIS: sin límite documentado, iNaturalist: 100 req/min). Inconsistencia de datos (especies en OBIS sin coincidencia en iNaturalist). Latencia acumulada por múltiples peticiones HTTP.\\
\textbf{Deuda técnica}&Ausencia de caching (cada escaneo regenera peticiones a APIs). Falta de paginación (actualmente limitado a 50 especies). Gestión de errores distribuidos no exhaustiva (si iNaturalist falla, se pierden datos visuales).\\
	\end{coolTable}
	\caption{Análisis de valor aportado: Escáner\label{tab:valorAportadoScan}}
\end{table*}

\section{Diseño}

He centrado el diseño en resolver tres incompatibilidades críticas entre el modelo de interacción del usuario y las restricciones de las APIs científicas:

\subsection{Incompatibilidad 1: Búsqueda radial vs. geométrica}

\begin{table*}[htb]
	\centering
	\begin{coolTable}{p{4cm}p{\textwidth-4.5cm}}{2}
{Memorando técnico 001: Geometría de Búsqueda}
\textbf{Asunto}&Incompatibilidad de búsqueda radial en API OBIS.\\
\textbf{Resumen}&Generación de polígonos WKT a partir de coordenadas centrales y radio.\\
\midrule
\textbf{Factores causantes}&El frontend solicita datos basados en un punto central \texttt{(lat, lng)} y un nivel de zoom (que determina el radio de visión), pero OBIS requiere un objeto \texttt{POLYGON} en formato WKT (Well-Known Text). La API no acepta parámetros \texttt{centerLat}, \texttt{centerLng}, \texttt{radius}.\\
\textbf{Solución}&He implementado \texttt{GeometryUtils.createPolygonFromRadius(lat, lng, radius)} que calcula los vértices de un polígono aproximadamente circular (o cuadrado simplificado) inscrito en el radio de visión. El resultado se serializa en formato WKT: \texttt{POLYGON((x1 y1, x2 y2, ..., x1 y1))}.\\
\textbf{Motivación}&OBIS es la mayor base de datos abierta de biodiversidad marina (>150M registros). Es imperativo adaptarse a su interfaz en lugar de buscar alternativas menores.\\
\textbf{Alternativas}&\textbf{Filtrado post-proceso}: Pedir un área muy grande (bounding box global) y filtrar en memoria por distancia euclidiana. Descartado por ineficiencia (transferencia de datos innecesaria) y riesgo de exceder límites de respuesta de OBIS (\texttt{size=10000} máximo).\\
\textbf{Cuestiones abiertas}&Actualmente uso un polígono cuadrado. Para zonas ecuatoriales, sería más preciso un polígono circular de N lados (N=12-16) para evitar incluir puntos fuera del radio real.\\
	\end{coolTable}
	\caption{Memorando técnico 001: Geometría}
\end{table*}

\textbf{Ejemplo de petición generada}:

\begin{verbatim}
https://api.obis.org/v3/occurrence?
  geometry=POLYGON((-4.0 36.5, -3.4 36.5, -3.4 37.0, 
                    -4.0 37.0, -4.0 36.5))
  &taxonid=2,3,4
  &size=50
  &fields=scientificName,decimalLongitude,decimalLatitude,
          eventDate,phylum
\end{verbatim}

\textbf{Parámetros clave}:
\begin{itemize}
    \item \texttt{geometry}: Polígono WKT cerrado (primer y último punto idénticos).
    \item \texttt{taxonid=2,3,4}: Filtro por phyla (2=Chordata, 3=Arthropoda, 4=Mollusca) para excluir microorganismos unicelulares (aunque no es exhaustivo).
    \item \texttt{fields}: Proyección de campos para reducir payload (omite metadatos innecesarios como \texttt{institutionCode}).
\end{itemize}

\subsection{Incompatibilidad 2: Calidad de datos visuales}

\begin{table*}[htb]
	\centering
	\begin{coolTable}{p{4cm}p{\textwidth-4.5cm}}{2}
{Memorando técnico 002: Calidad de Datos}
\textbf{Asunto}&Presencia de microorganismos y especies sin representación visual.\\
\textbf{Resumen}&Filtrado cruzado mediante existencia de recursos fotográficos en iNaturalist.\\
\midrule
\textbf{Factores causantes}&OBIS devuelve todo tipo de registros biológicos, incluyendo: \textbf{(1)} Microorganismos unicelulares (ej. \textit{Pycnococcaceae}), \textbf{(2)} Especies con nombre científico válido pero sin observaciones fotografiadas (ej. \textit{Thenea muricata}). Para una app turística/deportiva, estos datos no aportan valor experiencial.\\
\textbf{Solución}&Para cada nombre científico devuelto por OBIS, ejecutar una petición a iNaturalist: \texttt{GET /taxa?q=<scientificName>\&per\_page=1}. Aplicar los siguientes filtros: \textbf{(1)} Si \texttt{total\_results == 0}, descartar especie (no documentada en iNaturalist). \textbf{(2)} Si \texttt{default\_photo} es \texttt{null}, descartar especie (sin representación visual). \textbf{(3)} Si ambos criterios se cumplen, extraer \texttt{preferred\_common\_name} y \texttt{default\_photo.url} para enriquecer el modelo de respuesta.\\
\textbf{Motivación}&Mantener la relevancia visual y la experiencia de usuario. Un buceador no puede "ver" bacterias submarinas, por lo que incluirlas sería ruido informativo.\\
\textbf{Cuestiones abiertas}&El aumento de latencia por N peticiones HTTP a iNaturalist (N = número de especies únicas en OBIS). Para áreas con >50 especies, el tiempo de respuesta puede superar los 5 segundos. Posible optimización: batching de peticiones o caching de taxonomías previamente consultadas.\\
	\end{coolTable}
	\caption{Memorando técnico 002: Filtrado}
\end{table*}

\textbf{Ejemplo de especie descartada (microorganismo)}:

OBIS devuelve:
\begin{verbatim}
{
  "scientificName": "Pycnococcaceae",
  "phylum": "Cyanobacteria",
  ...
}
\end{verbatim}

iNaturalist responde:
\begin{verbatim}
{
  "total_results": 0,
  "results": []
}
\end{verbatim}

$\Rightarrow$ \textbf{Acción}: Especie filtrada, no incluida en respuesta al cliente.

\textbf{Ejemplo de especie válida con foto}:

OBIS devuelve:
\begin{verbatim}
{
  "scientificName": "Chimaera monstrosa",
  "phylum": "Chordata",
  ...
}
\end{verbatim}

iNaturalist responde:
\begin{verbatim}
{
  "total_results": 1203,
  "results": [{
    "preferred_common_name": "Rabbit Fish",
    "default_photo": {
      "url": "https://inaturalist-open-data.s3.amazonaws.com/..."
    }
  }]
}
\end{verbatim}

$\Rightarrow$ \textbf{Acción}: Especie enriquecida con foto y nombre común.

\subsection{Uso del campo \texttt{phylum} como fallback}

En caso de que iNaturalist devuelva \texttt{total\_results > 0} pero \texttt{default\_photo == null}, se usa el campo \texttt{phylum} (extraído de OBIS) para asignar un placeholder visual según taxonomía:

\begin{itemize}
    \item \texttt{Chordata} $\rightarrow$ Icono de pez genérico.
    \item \texttt{Mollusca} $\rightarrow$ Icono de pulpo/calamar.
    \item \texttt{Arthropoda} $\rightarrow$ Icono de crustáceo.
    \item Otros $\rightarrow$ Icono genérico de organismo marino.
\end{itemize}

\section{Implementación}

He implementado \texttt{SpeciesService.scanSpecies(lat, lng, radius)} que:

\begin{enumerate}
    \item Convierte coordenadas + radio a formato WKT POLYGON.
    \item Consulta OBIS con el polígono generado.
    \item Agrupa resultados por nombre científico (deduplicación).
    \item Para cada especie, consulta iNaturalist para obtener foto y nombre común.
    \item Descarta especies sin foto o sin documentación en iNaturalist.
    \item Retorna lista ordenada por número de ocurrencias.
\end{enumerate}

\textbf{Endpoint REST}: \texttt{GET /api/species?lat=\{lat\}\&lng=\{lng\}\&radius=\{radius\}}

\section{Pruebas}

He implementado los siguientes casos antes del código productivo:

\begin{itemize}
    \item \textbf{shouldGenerateValidWKTPolygon}: Verifica que, dado un punto \texttt{(36.5, -4.0)} y radio \texttt{5000m}, se genera un String \texttt{POLYGON((...))} geométricamente cerrado (primer y último punto idénticos).
    
    \item \textbf{shouldFilterMicroorganismsWithZeroResults}: Mockea una respuesta de OBIS con \textit{Pycnococcaceae} y simula \texttt{total\_results=0} en iNaturalist. Verifica que la lista final de \texttt{SpeciesResponse} está vacía.
    
    \item \textbf{shouldEnrichSpeciesWithPhoto}: Mockea OBIS devolviendo \textit{Chimaera monstrosa} e iNaturalist devolviendo foto + nombre común. Verifica que \texttt{SpeciesResponse.photoUrl} no es null y \texttt{commonName == "Rabbit Fish"}.
    
    \item \textbf{shouldHandleINaturalistTimeout}: Simula timeout en petición a iNaturalist. Verifica que el sistema descarta esa especie y continúa procesando las demás (resilencia).
\end{itemize}

\subsection{Documentación de API (TD1)}

He integrado Swagger/OpenAPI 3.0 con \texttt{springdoc-openapi-ui}:
\begin{itemize}
    \item \textbf{Interfaz interactiva}: Swagger UI en \texttt{http://localhost:8080/swagger-ui.html}
    \item \textbf{Especificación}: JSON/YAML generado automáticamente desde anotaciones
    \item \textbf{Validación}: Parámetros validados mediante anotaciones \texttt{@Parameter}
    \item \textbf{Respuestas}: \texttt{200 OK} con lista de especies, \texttt{400 Bad Request} para coordenadas inválidas
\end{itemize}

\subsection{Integración frontend (TD2)}

El frontend React consume el endpoint mediante:
\begin{itemize}
    \item \textbf{Componentes}: \texttt{MapView.tsx} (mapa interactivo), \texttt{SpeciesPanel.tsx} (resultados), \texttt{ScanningAnimation.tsx} (efecto sonar)
    \item \textbf{Estado}: MobX gestiona especies, carga y errores globalmente
    \item \textbf{Flujo}: Click en mapa → captura coordenadas/zoom → animación sonar → petición HTTP → actualización estado → renderizado panel
    \item \textbf{Validación}: Zoom mínimo 8 para evitar áreas excesivamente grandes
\end{itemize}

\subsection{Configuración del proyecto (TD3)}

\textbf{Backend (Spring Boot)}:
\begin{itemize}
    \item \textbf{Puerto}: 8080 para el servidor.
    \item \textbf{APIs externas}: URLs de OBIS, iNaturalist y Open-Meteo configuradas en \texttt{application.properties}.
    \item \textbf{Base de datos}: H2 en memoria (\texttt{jdbc:h2:mem:scubexdb}) para desarrollo rápido sin persistencia.
    \item \textbf{Seguridad}: CSRF deshabilitado (API REST), endpoints públicos (\texttt{/api/species}, \texttt{/swagger-ui}), OAuth2 pendiente para iteraciones futuras.
    \item \textbf{CORS}: Frontend local (\texttt{localhost:5173}) permitido para desarrollo.
\end{itemize}

\textbf{Frontend (React + Vite)}:
\begin{itemize}
    \item \textbf{Puerto}: 5173 con Hot Module Replacement.
    \item \textbf{Proxy}: Peticiones a \texttt{/api} redirigidas a \texttt{localhost:8080} (evita CORS).
    \item \textbf{Gestión de estado}: MobX para estado reactivo de especies y carga.
\end{itemize}

\section{Despliegue}

El despliegue del proyecto aún no se ha realizado. Actualmente se ejecuta en entorno de desarrollo local:

\begin{itemize}
    \item \textbf{Backend}: \texttt{mvn spring-boot:run} en puerto 8080.
    \item \textbf{Frontend}: \texttt{npm run dev} en puerto 5173.
\end{itemize}

En iteraciones futuras se planea despliegue en contenedores Docker y hosting en la nube.

\section{Seguimiento y control}

\textbf{Fase del proyecto}: Iteración 1 (Fases 3-5 de FDD).

\textbf{Estado}:
\begin{itemize}
    \item [OK] Endpoint \texttt{/api/species} implementado y testeado.
    \item [OK] Transformación geométrica (WKT Polygon) funcionando.
    \item [OK] Filtrado de microorganismos operativo.
    \item [OK] Enriquecimiento con iNaturalist completo (foto + nombre común).
    \item [ADVERTENCIA] Latencia observable en zonas con >30 especies (5-7 segundos). Candidata a optimización en Iteración 3 (sistema de caché).
\end{itemize}

\textbf{Métricas}:
\begin{itemize}
    \item \textbf{Cobertura de tests}: 87\% en \texttt{SpeciesService}.
    \item \textbf{Tiempo de respuesta (promedio)}: 3.2s para radio de 5km.
\end{itemize}

\textbf{Decisiones técnicas pendientes}:
\begin{itemize}
    \item Implementar caching de taxonomías previamente consultadas en iNaturalist (Redis o in-memory LRU cache).
    \item Añadir paginación al endpoint (\texttt{?page=1\&size=20}) para manejar áreas con alta biodiversidad.
\end{itemize}