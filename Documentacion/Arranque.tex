%!TEX root =  tfg.tex
\chapter{Arranque}

\begin{abstract}
Este capítulo define la arquitectura base del proyecto Scubex y la lista inicial de características priorizadas para el desarrollo del Producto Mínimo Viable (MVP), centrado en la exploración de biodiversidad marina.
\end{abstract}

\section{Lista de características}

Siguiendo la metodología de desarrollo, se han identificado las siguientes características nucleares:

\begin{enumerate}
    \item \textbf{Gestión de Usuarios (OAuth2)}: Autenticación delegada mediante Google.
    \item \textbf{Exploración Geográfica}: Visualización de mapa interactivo con datos climáticos.
    \item \textbf{Escáner de Especies}: Identificación de fauna marina en tiempo real basada en coordenadas.
    \item \textbf{Perfil de Usuario}: Persistencia de preferencias y datos básicos.
\end{enumerate}

\section{Diseño arquitectónico}
\label{sec:arquitectura}

El sistema sigue una arquitectura de cliente-servidor desacoplada.

\subsection{Stack Tecnológico}

\begin{itemize}
    \item \textbf{Backend}: Spring Boot 4 (Java 21). Actúa como Gateway de APIs científicas.
    \item \textbf{Frontend}: React 18 + Vite. SPA para una experiencia de usuario fluida.
    \item \textbf{Base de Datos}: H2 (Embebida) para datos de usuario; sin persistencia propia de datos biológicos (se consumen en tiempo real).
    \item \textbf{Seguridad}: Spring Security (OAuth2 Client).
\end{itemize}

\subsection{Integración de APIs Externas}

\begin{itemize}
    \item \textbf{OBIS}: Proveedor de ocurrencias biológicas (Nombre científico, Coordenadas, Taxonomía).
    \item \textbf{iNaturalist}: Proveedor de metadatos multimedia (Fotos, Nombres comunes).
    \item \textbf{WeatherAPI}: Datos meteorológicos (Viento).
    \item \textbf{Stormglass}: Datos oceanográficos (Temp. agua) [Uso restringido por rate-limit].
\end{itemize}
