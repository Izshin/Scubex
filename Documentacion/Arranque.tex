%!TEX root =  tfg.tex
\chapter{Arranque}

\begin{abstract}
Este capítulo define la arquitectura base del proyecto Scubex y la lista inicial de características priorizadas para el desarrollo del Producto Mínimo Viable (MVP), centrado en la exploración de biodiversidad marina y condiciones climáticas en tiempo real.
\end{abstract}

\section{Lista de características}

Siguiendo la fase 2 de FDD (Desarrollar lista de características), se han identificado las siguientes funcionalidades nucleares priorizadas por su valor técnico y experiencia de usuario:

\begin{enumerate}
    \item \textbf{Gestión de Usuarios (OAuth2 + H2)}: Autenticación delegada mediante Google y persistencia de perfil (nombre, foto, preferencias).
    \item \textbf{Exploración Geográfica}: Visualización de mapa interactivo mediante MapLibre GL JS con controles de zoom y navegación.
    \item \textbf{Escáner de Especies}: Identificación de fauna marina en una zona mediante radio de búsqueda y validación cruzada de APIs científicas.
    \item \textbf{Datos Climáticos y Oceanográficos}: Integración completa mediante Open-Meteo (viento, temperatura del agua, corrientes, olas, datos atmosféricos).
    \item \textbf{Red Social Geoespacial}: Sistema de posts geolocalizados con likes y comentarios (feature futura).
\end{enumerate}

\textbf{Justificación de priorización}:
\begin{itemize}
    \item OAuth2 se implementa primero para establecer el modelo de autenticación antes de añadir persistencia de datos de usuario.
    \item El escáner de especies, clima y publicación de avistamientos constituyen el MVP funcional mínimo que justifica el proyecto, pues no hay alternativas con interfaces intuitivas.
    \item El resto de funciones sociales son el punto de extensión para añadir valor al producto de manera incremental.
\end{itemize}

\section{Diseño arquitectónico}
\label{sec:arquitectura}

El sistema sigue una arquitectura de \textbf{cliente-servidor desacoplada} con Gateway de APIs científicas.

\subsection{Stack Tecnológico}

\begin{itemize}
    \item \textbf{Backend}: Spring Boot 4 (Java 21). Actúa como Gateway de APIs científicas, orquestador de peticiones asíncronas y gestor de autenticación.
    \item \textbf{Frontend}: React 18 + Vite. Con Hot Module Replacement (HMR) para desarrollo ágil.
    \item \textbf{Cartografía}: MapLibre GL JS. Motor de renderizado vectorial para animaciones fluidas y bajo consumo de datos.
    \item \textbf{Base de Datos}: H2 (Embebida) para datos de usuario (perfiles, posts, likes). Sin persistencia de datos biológicos (consumo en tiempo real desde APIs).
    \item \textbf{Seguridad}: Spring Security con OAuth2 Client para Google Login.
    \item \textbf{Documentación de API}: OpenAPI 3.0 (Swagger UI) con anotaciones automáticas.
\end{itemize}

\subsection{Justificación de MapLibre GL JS}

A diferencia de Google Maps API o Leaflet, MapLibre GL JS se ha seleccionado por:
\begin{itemize}
    \item \textbf{Sin necesidad de geocodificación}: El proyecto no requiere búsqueda de direcciones ni rutas, solo visualización de capas geoespaciales.
    \item \textbf{Open Source y sin rate limits}: Alternativa libre a Mapbox GL (del cual es fork).
\end{itemize}

\subsection{Integración de APIs Externas}

\begin{table*}[htb]
	\centering
	\begin{coolTable}{p{3.5cm}p{3.5cm}p{\textwidth-7.5cm}}{3}
{APIs externas integradas en Scubex}
\textbf{API}&\textbf{Proveedor}&\textbf{Datos proporcionados}\\
\midrule
OBIS&Ocean Biodiversity Information System& Ocurrencias biológicas: nombre científico, coordenadas, taxonomía (phylum), fecha de avistamiento.\\
iNaturalist&iNaturalist.org& Metadatos multimedia: fotos de especies, nombres comunes (preferred\_common\_name).\\
Open-Meteo&Open-Meteo.com& Datos climáticos y oceanográficos: viento, temperatura del agua/aire, altura de olas, corrientes marinas, precipitaciones. API gratuita sin necesidad de API key y plan gratuito generosísimo.\\
	\end{coolTable}
	\caption{Resumen de APIs externas}
\end{table*}

\textbf{Estrategia de consumo}:
\begin{itemize}
    \item \textbf{OBIS + iNaturalist}: Consultas bajo demanda mediante el botón "Scan Area" (evita rate limit innecesario).
    \item \textbf{Open-Meteo}: Polling cada 30 minutos para actualizar capas de viento y datos oceanográficos. Sin necesidad de API key y límites muy generosos (10,000 peticiones diarias en plan gratuito).
\end{itemize}

\subsection{Patrón Arquitectónico: Gateway API}

El backend actúa como \textbf{Backend for Frontend (BFF)}, orquestando múltiples APIs externas y aplicando lógica de negocio antes de exponer un contrato simplificado al cliente React:

\begin{enumerate}
    \item \textbf{Cliente React} invoca \texttt{GET /api/species?lat=X\&lng=Y\&radius=Z}.
    \item \textbf{SpeciesService} (backend) ejecuta:
    \begin{enumerate}
        \item Transformación geométrica: \texttt{(lat, lng, radius)} $\rightarrow$ \texttt{WKT POLYGON}.
        \item Petición a OBIS: \texttt{GET /occurrence?geometry=POLYGON(...)}.
        \item Para cada especie única devuelta por OBIS:
        \begin{itemize}
            \item Petición a iNaturalist: \texttt{GET /taxa?q=<scientificName>}.
            \item Filtrado: Si \texttt{total\_results == 0}, descartar especie.
            \item Enriquecimiento: Extraer foto y nombre común.
        \end{itemize}
    \end{enumerate}
    \item \textbf{Respuesta JSON} unificada con modelo \texttt{SpeciesResponse}.
\end{enumerate}

\textbf{Ventajas}:
\begin{itemize}
    \item El cliente React no necesita conocer la existencia de múltiples APIs.
    \item La lógica de filtrado (microorganismos, especies sin foto) se centraliza en el backend.
    \item Se evita exposición de API keys en el navegador.
\end{itemize}

\section{Seguimiento y control}

\textbf{Fase del proyecto}: Arranque (Fase 1 de FDD: Modelo Global).

\textbf{Estado}:
\begin{itemize}
    \item [OK] Arquitectura definida y documentada.
    \item [OK] Stack tecnológico seleccionado y justificado.
    \item [OK] APIs externas identificadas y contratos analizados.
    \item [PENDIENTE] OAuth2 pendiente de implementación (TB1 en TODO).
    \item [PENDIENTE] H2 Database pendiente de configuración.
\end{itemize}

\textbf{Decisiones arquitectónicas pendientes}:
\begin{itemize}
    \item Estrategia de caching para respuestas de OBIS (considerar Redis si el rate limit se vuelve problemático).
    \item Estrategia de despliegue (considerar Firebase?).
\end{itemize}
