%!TEX root =  tfg.tex
\chapter{Metodología}

\begin{abstract}
Este capítulo describe el marco metodológico adoptado para el desarrollo de Scubex, basado en Feature-Driven Development (FDD) con adaptaciones específicas para un proyecto académico individual. Se detallan las fases del ciclo de vida, la integración de Test-Driven Development (TDD) y el sistema de seguimiento iterativo empleado.
\end{abstract}

\section{Estructura organizacional del proyecto}

El proyecto Scubex ha sido desarrollado de forma individual por el autor como Trabajo Fin de Grado, asumiendo las responsabilidades de:

\begin{itemize}
    \item \textbf{Arquitecto de Software}: Diseño de la arquitectura cliente-servidor y selección del stack tecnológico.
    \item \textbf{Desarrollador Full-Stack}: Implementación del backend (Spring Boot) y frontend (React + Vite).
    \item \textbf{Integrador de APIs}: Orquestación de servicios externos (OBIS, iNaturalist, WeatherAPI, Stormglass).
    \item \textbf{Documentador Técnico}: Redacción de la memoria académica y especificación de APIs mediante OpenAPI (Swagger).
\end{itemize}

La supervisión técnica y académica ha sido proporcionada por el tutor del proyecto, quien ha validado decisiones arquitectónicas y avances iterativos.

\section{Metodología de desarrollo}

\subsection{Justificación de Feature-Driven Development (FDD)}

Se ha adoptado \textbf{Feature-Driven Development} como metodología central debido a sus características idóneas para proyectos de alcance medio con requisitos funcionales bien definidos:

\begin{itemize}
    \item \textbf{Orientación a características}: Cada funcionalidad (escáner de especies, integración climática, red social) se desarrolla como una unidad atómica documentable.
    \item \textbf{Iteraciones cortas}: Ciclos de 1-2 semanas permiten validar integraciones complejas con APIs externas de forma incremental.
    \item \textbf{Trazabilidad}: Mapeo directo entre requisitos funcionales, código y documentación académica.
    \item \textbf{Escalabilidad}: Facilita la adición de nuevas características (clima, oceanografía, social) sin refactorizar el núcleo existente.
\end{itemize}

A diferencia de Scrum (orientado a equipos) o Kanban (enfocado en flujo continuo), FDD proporciona un marco estructurado para proyectos donde la calidad del diseño y la documentación técnica son criterios de evaluación académica.

\subsection{Adaptación del ciclo FDD para Scubex}

El modelo clásico de FDD consta de 5 fases. Para este proyecto académico se ha adaptado la secuencia y se ha añadido una fase de documentación:

\begin{enumerate}
    \item \textbf{Construir un modelo global}: Diseño inicial de la arquitectura del sistema (capítulo de Arranque).
    
    \item \textbf{Desarrollar la lista de características}: Identificación y priorización de funcionalidades nucleares (Gestión de usuarios OAuth2, Escáner de especies, Exploración climática, Red social).
    
    \item \textbf{Planificar por característica}: Estimación temporal y asignación de recursos para cada iteración.
    
    \item \textbf{Diseñar por característica}: Especificación detallada mediante diagramas de colaboración y tablas de análisis de valor (Memorandos Técnicos).
    
    \item \textbf{Construir por característica}: Implementación incremental con validación mediante pruebas automatizadas (TDD).
    
    \item \textbf{Documentar la característica}: Redacción académica de cada iteración en LaTeX, incluyendo justificación de decisiones técnicas y lecciones aprendidas.
\end{enumerate}

\subsection{Integración de Test-Driven Development (TDD)}

Dentro de la fase 5 (Construir por característica), se ha aplicado \textbf{TDD} como práctica de ingeniería para garantizar la robustez de las integraciones con APIs externas:

\begin{table*}[htb]
	\centering
	\begin{coolTable}{p{4cm}p{\textwidth-4.5cm}}{2}
{Ciclo TDD aplicado en Scubex}
\textbf{Red}& Escribir una prueba que falle para el comportamiento esperado (Ej: \texttt{shouldFilterMicroorganisms}).\\
\midrule
\textbf{Green}& Implementar el código mínimo para que la prueba pase (Ej: Lógica de filtrado en \texttt{SpeciesService}).\\
\textbf{Refactor}& Mejorar el diseño sin cambiar el comportamiento (Ej: Extraer utilidades geométricas a \texttt{GeometryUtils}).\\
\textbf{Commit}& Versionar el cambio con mensaje descriptivo en Git.\\
	\end{coolTable}
	\caption{Ciclo Red-Green-Refactor aplicado}
\end{table*}

\textbf{Motivación de TDD en este contexto}:
\begin{itemize}
    \item Las APIs externas (OBIS, iNaturalist) tienen comportamientos impredecibles (inconsistencias de datos, cambios de formato).
    \item Las transformaciones geométricas (coordenadas → WKT Polygon) requieren validación matemática rigurosa.
    \item El filtrado cruzado de especies (OBIS $\times$ iNaturalist) introduce complejidad lógica que debe ser testeable de forma aislada.
\end{itemize}

\subsection{Herramientas de desarrollo}

\begin{itemize}
    \item \textbf{Control de versiones}: Git + GitHub (commits atómicos por característica).
    \item \textbf{Gestión de dependencias}: Maven (backend), npm (frontend).
    \item \textbf{Testing}: JUnit 5 + Mockito (backend), Vitest (frontend).
    \item \textbf{Documentación de API}: Swagger/OpenAPI 3.0 (generación automática de contratos REST).
    \item \textbf{Compilación}: Vite (desarrollo frontend con HMR), Spring Boot Maven Plugin (empaquetado JAR).
\end{itemize}

\section{Seguimiento y control}

El progreso del proyecto se ha monitorizado mediante:

\begin{itemize}
    \item \textbf{Archivo TODO}: Lista priorizada de tareas pendientes con urgencias marcadas (!).
    \item \textbf{Commits semánticos}: Mensajes estructurados (feat, fix, docs, refactor) para trazabilidad histórica.
    \item \textbf{Reuniones de seguimiento}: Sesiones quincenales con el tutor para validar decisiones arquitectónicas y resolver bloqueos técnicos.
    \item \textbf{Documentación incremental}: Redacción de capítulos de iteración inmediatamente tras completar cada fase de desarrollo.
\end{itemize}

\textbf{Métricas de avance}:
\begin{itemize}
    \item Número de características completadas (Iteración 1: Escáner, Iteración 2: Clima, etc.).
    \item Cobertura de tests unitarios (objetivo: >70\% en lógica de negocio).
    \item Páginas de documentación LaTeX redactadas por iteración (objetivo: 8-12 páginas/iteración).
\end{itemize}