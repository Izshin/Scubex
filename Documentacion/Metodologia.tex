%!TEX root =  tfg.tex
\chapter{Metodología}

\begin{abstract}
Este capítulo describe el marco metodológico adoptado para el desarrollo de Scubex, basado en Feature-Driven Development (FDD) con adaptaciones específicas para un proyecto académico individual. 
\end{abstract}

\section{Estructura organizacional del proyecto}

He desarrollado Scubex de forma individual como Trabajo Fin de Grado, asumiendo las responsabilidades de:

\begin{itemize}
    \item \textbf{Arquitecto de Software}: Diseño de la arquitectura cliente-servidor y selección del stack tecnológico.
    \item \textbf{Desarrollador Full-Stack}: Implementación del backend (Spring Boot) y frontend (React + Vite).
    \item \textbf{Integrador de APIs}: Orquestación de servicios externos (OBIS, iNaturalist, WeatherAPI, Stormglass).
    \item \textbf{Documentador Técnico}: Redacción de la memoria académica y especificación de APIs mediante OpenAPI (Swagger).
\end{itemize}

Mi tutor ha proporcionado la supervisión técnica y académica, validando decisiones arquitectónicas y avances iterativos.

\section{Metodología de desarrollo}

\subsection{Justificación de Feature-Driven Development (FDD)}

He adoptado \textbf{Feature-Driven Development} como metodología central debido a sus características idóneas para proyectos de alcance medio con requisitos funcionales bien definidos:

\begin{itemize}
    \item \textbf{Orientación a características}: Cada funcionalidad (escáner de especies, integración climática, red social) se desarrolla como una unidad atómica documentable.
    \item \textbf{Trazabilidad}: Mapeo directo entre requisitos funcionales, código y documentación académica.
    \item \textbf{Escalabilidad}: Facilita la adición de nuevas características (clima, oceanografía, social) sin refactorizar el núcleo existente.
\end{itemize}

En resumen, me permité, estructurar el desarrollo según mis preferencias, con una lista de caracterísiticas claras, priorizadas y empleando un lenguaje sencillo. Es adecuado para el caso, un proyecto hecho de forma individual
en el que mi estilo de desarrollo es el único relevante.
\subsection{Adaptación del ciclo FDD para Scubex}

El modelo clásico de FDD consta de 5 fases. Para este proyecto académico he adaptado la secuencia, unificado la fase de diseño y he añadido una fase de documentación:

\begin{enumerate}
    \item \textbf{Construir un modelo global}: Diseño inicial de la arquitectura del sistema (capítulo de Arranque).
    
    \item \textbf{Desarrollar la lista de características}: Identificación y priorización de funcionalidades nucleares (Gestión de usuarios OAuth2, Escáner de especies, Exploración climática, Red social).
    
    \item \textbf{Planificar por característica}: Estimación tecnológica y limitaciones y consecuente diseño del flujo de datos.
    
    \item \textbf{Construir por característica}: Implementación incremental con validación mediante pruebas constantes.
    
    \item \textbf{Documentar la característica}: Redacción académica de cada iteración en LaTeX, incluyendo justificación de decisiones técnicas y lecciones aprendidas.
\end{enumerate}

\subsection{Herramientas de desarrollo}

\begin{itemize}
    \item \textbf{Control de versiones}: Git + GitHub (commits atómicos por característica).
    \item \textbf{Gestión de dependencias}: Maven (backend), npm (frontend).
    \item \textbf{Testing}: JUnit 5 + Mockito (backend), Vitest (frontend).
    \item \textbf{Documentación de API}: Swagger/OpenAPI 3.0 (generación automática de contratos REST).
    \item \textbf{Compilación}: Vite (desarrollo frontend con HMR), Spring Boot Maven Plugin (empaquetado JAR).
\end{itemize}

\section{Seguimiento y control}

He monitorizado el progreso del proyecto mediante:

\begin{itemize}
    \item \textbf{Archivo TODO}: Lista priorizada de tareas pendientes con urgencias marcadas (!).
    \item \textbf{Commits semánticos}: Mensajes estructurados (feat, fix, docs, refactor) para trazabilidad histórica.
    \item \textbf{Reuniones de seguimiento}: Sesiones quincenales con el tutor para validar decisiones arquitectónicas y resolver bloqueos técnicos.
    \item \textbf{Documentación incremental}: Redacción de capítulos inmediatamente tras completar cada característica dentro de la aplicación.
\end{itemize}

\textbf{Métricas de avance}:
\begin{itemize}
    \item Características completadas.
    \item Cobertura de tests unitarios (objetivo: >70\% en lógica de negocio).
\end{itemize}